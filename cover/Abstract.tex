%!TEX root = ../Thesis.tex

\vspace*{20mm}

\noindent {\huge \textit{Abstract}}
\vspace{10mm}

Brain connectivity provides an anatomical substrate for brain functioning. Studies of brain network connectivity have determined a set of non-random organisational properties that are common across different species with varying scales of brain complexity. One of the most consistently identified property is the existence of a relatively small number of regions that form a disproportionately large number of connections to other areas of in the brain -- termed network hubs. Hubs also tend to be highly interconnected with each other, forming a so-called `rich-club' that plays a central role in integrated brain function. Rich-club organization has been identified at different scales including the microscale neuronal network of the \textit{C.elegans} worm, mesoscale connectomes of the mouse and macaque monkey as well as large-scale connectivity of the human brain. This strong conservation suggests the potential role of genetic influences in driving such organisation. Therefore, the thesis examines the genetic markers of brain network organisation in both micro- and macro- scale neural networks particularly focusing on the connectivity between brain network hubs. 

The first experimental chapter presents an investigation of gene expression properties in the microscale connectome of the \textit{C.elegans} worm. Employing a rich set of publicly available data we show that consistently with the previous findings in the mouse brain, hub neurons in the \textit{C.elegans} connectome display tightly coupled gene expression. We also demonstrate that this gene expression similarity cannot be attributed to a wide range of anatomical and topological properties including their neuronal subtype, spatial separation distance, chemically secreted neurotransmitter, birth time, pairwise lineage distance or topological module affiliation and instead is related to the functional role of most hubs as command interneurons -- a separate class of neurons that are responsible for guiding locomotion -- one of the most complex behaviours of the animal. 

The second experimental chapter addresses the methodological considerations involved in combining gene expression from the Allen Human Brain Atlas with imaging data. The emergence of brain-wide gene expression atlases opened new opportunities for investigating the relationships between the molecular function and macroscopic neuroimaging phenotypes, however the approaches for integrating those data modalities are not consistent across the literature posing challenges for reproducibility. Here we for the first time provide a seven-step processing pipeline for combining those data modalities and evaluate the influences of different methodological choices at each step, therefore making the first move towards a standardized analysis pipeline that can be routinely employed throughout the field. 

The last experimental chapter is dedicated to comprehensively investigate the genetic markers of hub connectivity in the large-scale connectome of the human brain. Combining diffusion weighted imaging, genotyping and brain-wide gene expression data we demonstrate that connections between hub regions are under the strongest genetic control, that genes contributing to the inter-individual variability in hub connectivity strength are implicated in mental illness and high IQ and that consistently with the previous findings in mouse and \textit{C.elegans} hub regions in the human brain possess a distinct transcriptional signature that is primarily driven by metabolism-related genes.

Together these findings demonstrate a direct link between molecular function and brain network organisation across different scales and indicate that the genetic influences on brain network organisation converge on highly connected and functionally important brain network hubs. 

