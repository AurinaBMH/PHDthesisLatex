%!TEX root = ../Thesis.tex

\vspace*{20mm}

\noindent {\huge \textit{Abstract}}
\vspace{10mm}

Brain connectivity provides an anatomical substrate for integrated brain function. Studies of brain network connectivity have uncovered a set of non-random organisational properties that are common across different species with varying scales of brain complexity. One of the most consistently identified properties is the existence of a small number of neural elements that form a relatively large number of connections. These elements are called network hubs. Hubs also tend to be highly interconnected with each other, forming a so-called `rich-club'. Rich-club organization has been identified at different scales including the microscale neuronal network of the \textit{C.elegans} worm, mesoscale connectivity of the mouse and macaque monkey, as well as macroscale human brain networks. This strong conservation suggests the potential role of genetic influences in driving such organisation. The evidence to support this claim are relatively sparse mostly coming from scattered studies with differing methodologies. This thesis examines genetic markers of brain network organisation in both micro- and macro- scale neural networks, particularly focusing on the connectivity between brain network hubs. Building on the previous work in mouse brain where hubs have been shown to demonstrate increased gene expression similarity, we first set to examine how gene expression patterns relate to neuronal connectivity in a microscale neuronal network of the \textit{C.elegans} -- an organism with a fully mapped nervous system at the level of neurons and synapses. Next, we focused on systematically examining the genetics of hub connectivity in the macroscale network of the human brain by integrating diffusion weighted imaging, brain-wide gene expression, and genotyping data.

The first experimental chapter presents an investigation of gene expression properties in the microscale connectome of the \textit{C.elegans}. Employing a rich set of publicly available data we show that, consistent with the previous findings in the mouse brain, hub neurons in the \textit{C.elegans} connectome display tightly coupled gene expression. This gene expression similarity cannot be attributed to a wide range of anatomical and topological properties including their neuronal subtype, spatial separation distance, chemically secreted neurotransmitter, birth time, pairwise lineage distance or topological module affiliation and instead is related to the functional role of most hubs as command interneurons -- a specific class of neurons that are responsible for guiding locomotion -- one of the most complex behaviours of the animal. Following this work, the next chapter provides a literature review of recent studies investigating gene expression correlates of hub connectivity across different scales. 

The second experimental chapter addresses methodological considerations involved in combining gene expression from the Allen Human Brain Atlas with neuroimaging data. The emergence of brain-wide gene expression atlases has provided new opportunities for investigating the relationships between the molecular function and macroscopic neuroimaging phenotypes, but approaches for integrating these data modalities are not consistent across the literature, posing major challenges for reproducibility. We provide a seven-step processing pipeline for combining those data modalities and evaluate the influences of different methodological choices at each step, therefore making the first move towards a standardised analysis pipeline that can benefit the reproducibility in the field. 

The last experimental chapter is dedicated to a comprehensive investigation of the genetic markers of hub connectivity in the macroscale connectome of the human brain. Combining diffusion weighted imaging, genotyping and brain-wide gene expression data, we demonstrate that connections between hub regions are under the strongest genetic control, that genes contributing to the inter-individual variability in hub connectivity strength are implicated in mental illness and high IQ. Moreover, consistently with the previous findings in mouse and \textit{C.elegans} hub regions in the human brain possess a distinct transcriptional signature that is primarily driven by metabolism-related genes.

Together these findings demonstrate a direct link between molecular function and brain network organisation across different resolution scales and species, and indicate that the genetic influences on brain network organisation converge on highly connected and functionally important brain network hubs. These findings are in line with the idea that hub connectivity, that is supporting integrative brain function, might be driven by common evolutionary principles and suggest the involvement of both metabolic and disease-related genes. Further investigations into the genetic basis of brain connectivity might shed some light onto the molecular mechanisms associated with psychiatric illness.