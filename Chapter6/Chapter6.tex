%!TEX root = ../Thesis.tex
\chapter{General discussion and concluding \\ remarks}
\label{ch:Discussion}
\fancyhead[R]{\textit{Chapter.} \textit{\thechapter: }\textit{General discussion and concluding remarks}}
%\fancyhead[ER]{\bf Ch. \thechapter~INTRODUCTION \& OVERVIEW}

\section{General discussion}
The overarching aim of this thesis was to investigate the molecular basis of large-scale brain connectome organization, with a particular focus on the genetic underpinnings of brain network hub connectivity. The critical role that hubs are assumed to play in cognitive performance \citep{Buckner2009,Mesulam1998}, coupled with the susceptibility of these regions to a range of brain disorders \citep{Bassett2009a,Crossley2014,Fornito2015} highlights their importance for integrated brain function. The striking conservation of network hubs and rich-club organization across species and scales together with the consistently high centrality and cost of these network elements \citep{VandenHeuvel2016}, suggest that hub interconnectivity may have evolved to satisfy competitive selection criteria of minimising connection costs while maintaining integrated communication between neural elements \citep{Bullmore2012}. Understanding the genetic basis of hub connectivity could thus shed light on key evolutionary principles for brain networks, the molecular basis for higher-order, integrative brain function, and genetic liability for brain disease.

This thesis examined the link between genetics and hub connectivity at the microscale of cells and synapses in \textit{C.elegans} and the macroscale of brain regions linked by white matter bundles in the human. As part of the latter analysis, a detailed pipeline for relating human neuroimaging and gene expression data was developed. Specific details of the findings of each experiment are detailed in each experimental chapter. The following sections outline some general implications of the major findings.

\subsection*{Hub connectivity is associated with a highly conserved transcriptional signature}

As previously demonstrated in the mesoscale mouse connectome \citep{Fulcher2016}, Chapter II showed that hub neurons in the \textit{C.elegans} connectome display an increased gene expression similarity. Findings presented in Chapter V demonstrate that a similar increase in transcriptional coupling between hub regions is also evident in the macroscale human connectome. This consistency is impressive given the large differences in resolution scales (worm: cells and synapses; mouse: cell populations and axons; human: brain regions and white matter bundles), connectome mapping methods (worm: electron microscopy for; mouse: tract-tracing; human: DWI), and gene expression quantification (worm: binary expression indicators derived from the previous literature; mouse: cellular resolution ISH; human: bulk tissue microarray). Both the mouse and human analyses indicated that the strongest contributions to the elevated CGE for pairs of hubs came from genes regulating oxidative metabolism. This specific result could not be replicated for \textit{C.elegans}, as expression data for such genes were not available.

A key aspect of this apparent transcriptional signature of hub connectivity is that it counters the more general trend in the brain, in which CGE and connection probability decline as a function of distance between regions. In contrast, hubs have higher CGE and are more likely to be connected despite being, on average, separated by longer physical distances. This distinctive increase in CGE between hub regions could arise from multiple sources including cellular and microcircuitry properties of hub regions. For example, primate studies have identified that hub regions have lower neural cell density \citep{Beul2017,Scholtens2014}. Higher levels of regional connectivity have also been associated with a range of microscopic properties indicating increased neuronal complexity including more elaborate pyramidal dendritic branching, higher total spine count and larger soma size \citep{Scholtens2014}. Moreover, regional cytoarchitectural properties have been related to their pairwise laminar-projection profiles linking local microcircuits to their hierarchical connectivity \citep{Barbas}. The cell-specific work in \textit{C.elegans} would argue against a simple cell density explanation indicating the important role of specific cell types. The analysis of gene expression in \textit{C.elegans} showed that functional identity of hub neurons was a significant factor in driving the increase in CGE between them. Future work investigating the cell-specific transcriptional properties of hub regions would be required in order to test this hypothesis in higher order animals.

\subsection*{Hub connectivity of the human connectome is under tight genetic control and is linked to genes for intelligence and mental illness}

Relating imaging to gene expression data provides a way for identifying how spatial patterns of gene expression track spatial variations of network phenotypes. However, they do not offer insight into whether or which genes regulate individual differences in hub connectivity. Chapter V thus combined two other commonly used methods for characterizing genetic influences on a given trait -- heritability analysis of twin data and polygenic score analysis of structural DNA variants.

The heritability analysis revealed that individual differences in the hub connectivity microstructure are under strong genetic control; in fact, the average heritability for rich links was significantly higher compared to feeder and peripheral links. Moreover, feeder connections also showed significantly higher heritability compared to peripheral links therefore demonstrating the graded nature of the effect where genetic control is increasing with increasing topological centrality of the connections. This result indicates that genes play a major role in shaping connectivity microstructure that appears to be highly selective for the connections between connectome hubs. This high heritability of hub connectivity is consistent with the hypothesis that the apparent trade-off between wiring cost and adaptive function may be strongly concentrated on the arrangement of hub connections \citep{Bullmore2012}. That is, genetic influences on the integrity of hub connectivity may influence the efficacy with which different brains negotiate this trade-off.

The genetic primacy of hub connections is supported by developmental considerations [for a review see \citep{Oldham2018}]. In \textit{C.elegans}, hub neurons are born at the earliest stages of the development \citep{Varier2011}, prior to evidence of coordinated movement in the animal. Structural hubs within the human brain networks also tend to emerge early in development, with studies demonstrating the presence of hubs in superior frontal, superior parietal and occipital regions as early as 30 weeks of gestational age \citep{Ball2014}. Hubs in preterm infants also form a highly interconnected rich-club, with most changes in the last 10 weeks of gestation occurring between hubs and other regions, rather than concentrating on the relatively mature rich connections between hubs \citep{Ball2014}. Furthermore, only minor changes in the connectivity strength between the putative core elements of the brain have been observed during the third trimester of fetal development \citep{Batalle2017} indicating the early formation and relative consistency of hub connections in the human brain. Thus, the basic binary topology of hub connectivity may be established early, but these connections may undergo a protracted period of development, resulting in changes to the weighted topology of the network \citep{Baker2015a,Oldham2018}.

The PGS analysis revealed that genetic liability for a range of psychiatric disorders is preferentially related to reduced connectivity between hub regions. At the same time, higher genetic scores for IQ showed the opposite relationship indicating that, on average, subjects with higher genetic scores for IQ have increased connection strength between hubs. This suggests that genes involved in driving the interindividual differences in hub connectivity are implicated in a broad range of phenotypes including mental illness and general intelligence. Such relationship converging on connections between hubs that are known to form a backbone for brain communication \citep{Harriger2012, VandenHeuvel2011,VandenHeuvel2013b} might be related to the differences in the integrative capacity of the brain that is linked to the increased genetic predisposition to psychiatric illness.

\subsection*{Standardized workflows are required for relating gene expression and \\neuroimaging data}

The capacity to link brain imaging data to gene expression measures is a recent innovation, possible only with the advent of gene expression atlases in different species [\citep{Harris2010,Hawrylycz2012,Lein2007a}, for a review see \citep{Keil2018}]. Although a growing number of studies are bridging the two types of data, there are no standard practices in the field and many data processing options are available. Chapter IV conducted the first systematic survey of such options and found that processing choices can significantly affect the resulting data. For example, some procedures that are important for the meaningful biological interpretation of the results are often ignored in the literature including the probe-to-gene re-annotation and the appropriate investigation of the spatial properties of gene expression. Whereas most of the processing choices require further consideration and should be tailored to particular research questions, it is necessary to ensure the implementation of comparable processing pipelines that would allow to draw links between findings across different studies. Moving forward, it will be important to work towards unified data processing methods when integrating brain-wide gene expression atlases with other modalities across species.

\section{Limitations and future directions}

Despite the consistency of genetic influences on hub connectivity identified in this thesis, each study has a number of limitations, particularly with respect to how some of the data were curated. The connectome data in \textit{C.elegans} is as close to a gold standard as possible in the field, representing the (near-)complete neuron-and-synapse wiring diagram of the organism's nervous system. However, the gene expression data was relatively incomplete, relying on the curated results stored in Wormbase \citep{Harris2010}. Being limited to the information acquired from the previous literature, such data aggregation methods are heavily biased in their coverage and do not allow to distinguish between the cases of missing information and absent gene expression. These drawbacks could be overcome by implementing single cell RNA sequencing to acquire genome-wide expression signatures for individual neurons, however these experiments pose a number of technical challenges including the implementation of advanced cell sorting techniques. Single-cell RNA sequencing has been performed on a larval stage of \textit{C.elegans} development \citep{Cao2017}, however it has not yet been extended to single-neuron gene expression in the adult worm.

The human gene expression data provide better genomic coverage and reasonable spatial coverage of one hemisphere of the brain, but the data are derived from 6 donor brains precluding a detailed investigation of individual differences in gene expression patterns \citep{Arnatkeviciute2019,Hawrylycz2015}. Moreover, as the expression measures are quantified in bulk tissue samples, they may be influenced by cellular composition of each sample \citep{Tasic2016}. Although microarray technique provides a cost-effective way of quantifying gene expression in a high-throughput manner, improvements in single-cell profiling should yield more detailed and informative gene expression maps \citep{Lein2017}. Considering that a very high proportion of genes demonstrates differential expression across development \citep{Colantuoni2011}, the generation of developmentally-resolved brain-wide atlases would be informative in uncovering the relationship between the transcriptional dynamics and brain connectivity.

The analyses of large-scale connectivity in the human brain presented in this thesis rely on the diffusion weighted imaging. While it is currently the only available technique allowing to noninvasively explore the structural connectivity of the human brain, the results derived from such indirect approaches should be interpreted with caution \citep{Jones2013,Sotiropoulos2017}. Tractography algorithms used to infer white matter pathways from the underlying diffusion signal have known biases, including difficulties in reconstructing long-range pathways \citep{Jones2010} mainly arising due to the high proportion of voxels containing crossing fibers \citep{Jeurissen2001}. Compared to deterministic algorithms, the usage of probabilistic tractography can at least partially mitigate this issue \citep{Tournier2010}, however, the lack of ground truth measures of brain connectivity makes it difficult to balance the sensitivity and specificity of the resulting connectivity estimates \citep{Zalesky2016}. Additional challenges arise in interpreting the DWI-derived measures. For example, changes in fractional anisotropy are frequently attributed to inherent differences in white matter integrity, however they can result from a number of other factors including differences in axon diameter, packing density or a number of crossing fibers in the area \citep{Jones2013,Takahashi2002}. These factors cannot be disentangled using current techniques, therefore, the ongoing development of more quantitative estimates of DWI-derived anatomical connectivity \citep{Bouhrara2016,Zhang2012} may prove useful in the future.

\newpage
\section{Conclusions}

The findings presented in this thesis contribute towards the rapidly developing field of imaging genetics by demonstrating consistent evidence for a strong genetic influence on hub connectivity. Integrating a range of data modalities across different species and scales, this work shows that the genetic influences on brain connectivity converge on topologically central and functionally important brain network hubs. More specifically, hubs tend to display increased transcriptional similarity in both micro- and macros-scale brain networks. Moreover, the connection strength between hub regions in the human brain is strongly heritable and is related to polygenic liability for mental illness and high IQ. The methodological findings on integrating brain-wide gene expression and neuroimaging data highlight the importance of ensuring consistency and reproducibility in atlas-based imaging genetics research.
