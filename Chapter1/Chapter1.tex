%!TEX root = ../Thesis.tex
\chapter{Introduction}
\label{ch:Introduction}
\fancyhead[R]{\textit{Chapter.} \textit{\thechapter: }\textit{Introduction}}
%\fancyhead[ER]{\bf Ch. \thechapter~INTRODUCTION \& OVERVIEW}

Brain connectivity forms the anatomical substrate for our consciousness, thoughts, experiences and emotions. Attempts to understand how the network of connections comprising the brain is organised began over a century ago by visualising individual neurons and observing their connectivity patterns. Recent developments in brain imaging techniques have enabled researchers to comprehensively examine neuronal connectivity across the entire brain, with a major focus of $21^{st}$ century neuroscience being on uncovering key properties of brain network organization in both health and disease \citep{VandenHeuvel2010b,Bullmore2012,Fornito2015}. 

Understanding the factors that sculpt brain network architecture is essential for future developments in both theoretical and clinical neuroscience. Brain connectivity, like most phenotypes, is the product of myriad environmental and genetic influences. The precise degree to which genes or environment influence distinct aspects of neural structure and function remains a topic of ongoing investigation. The past two decades of brain network research have identified a series of non-trivial properties that are highly conserved across species and resolution scales, from the level of individual neurons and synapses to the macroscale organization of the entire brain, suggesting that such connectivity patterns result from common selection pressures which may, at least in part, be encoded in the genome. 

This thesis focuses on understanding genetic contributions to one such property--connectivity between network hubs. Hubs are regions characterised with a high degree of connectivity to other areas. They also tend to preferentially connect to each other forming a putative core of the network and thus playing a central role in brain communication. Disruptions in connectivity and functioning of hub regions have been associated with a range of neurological and psychiatric disorders, therefore studying the genetic properties of hub connectivity can help uncover the underlying mechanisms behind brain network organisation in both health and disease. The remainder of this chapter provides a brief overview of the current research in brain connectivity and the main approaches used to investigate the genetic contributions towards the brain organisation, before outlining the general structure of the thesis. 


\section{Brain as a network}
A human brain contains around $86$ billion neurons connected by around $100$ trillion synapses \citep{Williams1988,Andersen1992,Pelvig2008} forming one of the most complex systems that we know. It is now universally accepted that the brain is comprised of discrete cells interconnected by synaptic junctions--the so-called neuron doctrine, first proposed in 1894 by Santiago Ram\'{o}n y Cajal \citep{RamonyCajal1995} and later confirmed in the 1950s with the development of the electron microscopy \citep{DeRobertis1955}. This doctrine established a foundation for the modern understanding of the nervous system as a network of discrete, interconnected elements. 

Another important observation by Ram\'{o}n y Cajal was that extraordinary complexity of neuronal organization may be explained by some simple underlying principles. Specifically, he proposed that: 
\bigskip
\bigskip

\textit{``… all of the various conformations of the neuron and its various components are simply morphological adaptations governed by laws governed by laws of conservation for time, space and material''}

\hspace{7.5cm}Santiago Ram\'{o}n y Cajal (1995), p.116, Volume I. 
\bigskip

Under these laws, neurons and their processes are organized in order to minimize axonal wiring cost, saving both cellular material and intracranial volume, and to minimize conduction delays in the transmission of information between neurons, which requires an efficient configuration of network connectivity. Numerous studies since have supported an important role for conservation of space and material, often operationalized as axonal wiring cost, showing that it can explain microscale properties such as the geometry of neuronal arbors \citep{Cherniak1999} and axonal connection volume \citep{Chklovskii2002} as well as macroscale features such as the placement of cortical areas \citep{Cherniak2004}. However, due to limitations on available data, these studies only focused on restricted sections of tissue or incomplete connectivity, which offers only partial insight into general principles of brain network organization.  

Developments in microscopy, axonal tract tracing, and non-invasive imaging techniques have led to the emergence of increasingly more precise and detailed connectivity maps in different species \citep{Bota2015,Chiang2011a,Oh2014,Scannell1999,Shanahan2013,Stephan2001,VanEssen2013,White1986}. Such maps range from the microscale connectivity at the level of individual neurons and regional connectivity at the mesoscale to the large-scale brain connectivity of the human brain. The creation of the first detailed wiring diagrams of the whole nervous systems started with serial electron microscope reconstructions of the hermaphrodite nematode \textit{C.elegans} \citep{White1986} later supplemented by additional experiments \citep{Varshney2011} resulting in the first full neuronal connectivity network defined at the levels of individual neurons and synapses. Consisting of $279$ neurons connected by $890$ gap junctions, $6393$ chemical synapses and $1410$ neuromuscular junctions, the somatic nervous system of the \textit{C.elegans} serves as a model network for the investigation of the microscale connectivity. In addition to the relatively simple wiring diagram of the \textit{C.elegans}, detailed interregional connectivity maps of the Drosophila fly brain have been constructed based on the neuronal connectivity of approximately $12000$ neurons resulting in a first representative neuronal map of a more complex brain \citep{Chiang2011a}.

Recently, the investigations have been expanded to the full connectivity maps of the larvae nervous system \citep{Ohyama2015} and the complete volume of the adult Drosophila brain \citep{Zheng2018}. Reconstructing whole-brain connectivity in larger animals predominantly involve tracing axonal projections between brain areas. Using such methods, inter-regional connectivity maps have been constructed for a range of species including pigeon \citep{Shanahan2013}, mouse \citep{Oh2014}, cat \citep{Scannell1995}, and macaque \citep{Harriger2012}. Investigations of the large-scale connectivity within the human brain rely on non-invasive imaging techniques. Combining diffusion weighted imaging (DWI) together with tractography algorithms \citep{Jbabdi2011} enable the reconstruction of the white matter pathways linking macroscopic brain regions. Despite uncertainties associated with the interpretation of the diffusion MRI signal and limitations in tract reconstruction \citep{Jones2010,Jones2013,Thomas2014}, DWI is currently the most widely adopted technique to assess structural brain connectivity human. 

Across these diverse species, measurement techniques and spatial resolutions, the resulting comprehensive maps of connectivity are called connectomes, and are typically represented as matrices displaying all pairwise anatomical connections between neural elements (e.g., neurons, cell populations, regions) of the brain \citep{Sporns2005}. The emerging field of connectomics, focused on understanding system-wide patterns of network structure that combined with the dynamics has yielded a more complete picture of the brain suggesting that a simple cost minimization principle alone may be insufficient to explain neuronal organization.

\section{Topological organisation of the brain}

A central tenet of connectomics is that the brain, like any other network, can be represented as a network of nodes (elements in a network) connected by edges (connections between elements). This general, yet comprehensive way of rendering a network allows investigation of both connectivity (i.e., the presence or absence of connections) and topology (i.e., the precise arrangement of connections between nodes or, more formally, properties of the network that are invariant to its layout in physical space) using the mathematics of graph theory \citep{Barabasi2016}.

Initial studies of brain network topology showed that the brain resembles a small-world architecture \citep{Bassett2006,Gygi1999,Hilgetag2004,Sporns2004,Watts1998} that is characterized by highly clustered, tightly connected subsets of nodes similar to a lattice, combined with sparse long-range projections. Those long distance connections have been implicated in reducing the topological distance within the network \citep{Bullmore2012,Sporns2004,VandenHeuvel2011} and demonstrated to increase network robustness and promote complex brain dynamics \citep{Betzel2018}. Several other studies have since shown a range of more complex topological properties including a hierarchical community structure of densely connected sub-networks, called modules, such that modules contain nested sub-modules over several resolution scales \citep{Bullmore1997,Meunier2010a,Towlson2013} that are thought to support functional specialization. Furthermore, connectivity across nervous systems is non-uniformly distributed, such that a relatively small number of nodes possess a disproportionately high number of connections, thus representing network hubs
\citep{Towlson2013,VandenHeuvel2011}.

The emergence of many of these properties can be explained in the context of Cajal’s conservation laws \citep{RamonyCajal1995}. For example, the tendency of neural elements to form connections with spatially proximal neighbours, resulting in anatomically localised communities, is consistent with a cost minimisation rule: geometric constraints of the system favour the formation of short-range connections facilitating the emergence of functionally specialised processing units while penalising biologically expensive long-range projections. In fact, simple generative brain network models based on the exponential decay rule have shown success in recreating networks many topological features that mimic the brain \citep{Ercsey-Ravasz2013,Henderson2014}. The only rule in such models states that connection probability is exponentially decreasing with increasing separation distance between network elements. However, several considerations indicate that this constraint alone is insufficient to explain brain connectivity, as long-range projections occur more frequently than predicted by a simple distance-dependent wiring rule. Introducing additional terms that balance cost minimization with a preference for certain adaptive topological properties, such as forming links between elements that connect to similar nodes, leads to a more accurate model of connectome organization \citep{Betzel2016,Vertes2012}. Such results are in line with a trade-off initially implied by Cajal’s conservation laws, in that the pressure to minimize wiring costs must be balanced with the need to promote functionally adaptive topology \citep{Bullmore2012}. 

\section{The role of brain network hubs}

Hub connectivity appears to play a major role in how the brain negotiates the trade-off between cost minimization and adaptive topology. Hubs are usually identified using a measure called node degree, which quantifies the number of connections attached to a particular node. In addition to being highly connected, hubs also demonstrate denser interconnectivity between each other than expected by chance forming a so-called``rich-club” \citep{Fulcher2016,Harriger2012,Towlson2013,VandenHeuvel2011,Zamora-Lopez2010}. By linking hubs that are distributed throughout the brain, the rich-club forms a biologically costly backbone of the connectome that mediates a high proportion of communication traffic \citep{VandenHeuvel2012}. Rich-club organization is conserved across different species and resolution scales, including the neuronal connectome of the \textit{C.elegans} \citep{Towlson2013}, the mesoscale connectomes of the mouse \citep{Oh2014,Fulcher2016}, cat \citep{DeReus2013b}, macaque \citep{Harriger2012}, and the macroscale connectivity of the human brain \citep{VandenHeuvel2011}, suggesting that it arises from common selection pressures operating on these nervous systems. 

As the connections between hubs extend over long anatomical distances \citep{Fulcher2016,Towlson2013,VandenHeuvel2011}, their judicious placement is critical to ensure that the substantial biological costs are offset by the functional advantages they confer \citep{DeReus2014,Towlson2013}. Consistent with this view, hubs of the human brain typically reside in precuneus, insular, anterior and posterior cingulate, superior frontal, lateral parietal and temporal cortices \citep{Gong2009,VandenHeuvel2012,VandenHeuvel2013a} -- areas of paralimbic and association cortex that are known to mediate complex brain function \citep{Buckner2009,Mesulam1998}. Individual differences in hub connectivity are linked to intelligence \citep{Li2009,VandenHeuvel2009}, cognitive task performance \citep{Cole2012}, and personality traits \citep{Adelstein2011}. Furthermore, axonal connections between hub regions have been shown to occupy high white matter volume and exhibit high levels of white matter organisation as quantified by increased fractional anisotropy, a measure that is thought to reflect relative axonal fiber density, diameter and myelination in white matter \citep{Collin2014}. These properties might indicate several advantages associated with increased communication efficiency and robustness along those paths \citep{Collin2014}. Increased involvement of hubs in cognitive processing \citep{Buckner2009,Cole2012,Mesulam1998} together with cost-related connectivity attributes are likely to impose increased metabolic demands on those regions \citep{Collin2014}. Indeed, studies of cerebral blood flow and metabolic activity indicate that hub regions to be among the most metabolically active areas \citep{Vaishnavi2010,Varkuti2011}. 


As argued by Bullmore and Sporns \citep{Bullmore2012}, it is intuitive to expect that an impact on a topologically central area would result in more widespread damage as affected hubs have a disproportionate effect on the integrative capacity of the brain \citep{DeReus2014}. Converging evidence suggests that hubs are more susceptible to a diverse range of disease [for a review see \citet{Bassett2009a,Crossley2014,Fornito2015b}]. The topologically central position of hubs means that they can be easily reached by a disruption originating elsewhere in the brain \citep{Zhou2012}. Moreover, many connections involving hubs extend long distances \citep{VandenHeuvel2012}, hence increasing their susceptibility to injury. For instance, schizophrenia patients demonstrate both structural and functional changes including reduced frontal hub connectivity \citep{Fornito2012a,VandenHeuvel2010,Zalesky2011} and disrupted rich-club formation \citep{VandenHeuvel2013c}. Aging studies have also identified hub connectivity changes in Alzheimer’s disease \citep{DeHaan2012,Stam2009} and frontotemporal dementia \citep{Agosta2013}. Moreover, changes in hub organisation \citep{Achard2012} and metabolic activity \citep{Laureys2004} have been associated with reduced consciousness states. Together these findings demonstrate the involvement of hubs in a broad range of disorders and highlight their importance in normal brain functioning. 

The emergence of hubs and rich-club connectivity across multiple species points to a strong conservation across the evolutionary tree, which may result from a general selection pressure to maintain adaptive performance at low metabolic cost \citep{Bullmore2012}. A corollary of this view is that hub connectivity may be under genetic control. Given the importance of hubs in brain function, an investigation into the genetics of hub connectivity is likely to have important implications for understanding the molecular basis of both behaviour and disease. 
\section{Genetic markers of brain connectivity}

Genetic influences on human brain networks have typically been examined using one of three primary methods: quantifying the degree of genetic control over a trait through heritability analysis, identifying associations between trait variability and allelic variation of structural DNA at the level of single nucleotide polymorphisms (SNPs), or investigating how anatomical variations of a given network property correlate with spatial patterning of gene expression. 

\subsection{Heritability}

Twin studies offer a natural experiment that allows researchers to quantify the heritability of a trait; i.e., the proportion of trait variance that is attributable to genes. Twin studies rely on the assumption that monozygotic (MZ) twins share all of their genetic information whereas dizygotic (DZ) twins on average have $50\%$ of their genes in common. Under the further assumption that environmental contributions affect both twin groups to a similar extent, structural equation modelling can be used to partition phenotypic variability into proportions of variance explained by common environment (non-genetic influences shared between members of a twin pair), unique environment (person-specific experiences and measurement error), and genetic influences. Generally, overall heritability of the trait entails the effect of additive, dominant and genetic interaction contributions, together termed the broad-sense heritability. In the context of imaging genetic studies, the heritability estimates traditionally are modelled to reflect a proportion of the total variance that is attributable to the additive allelic effects, termed the narrow sense heritability. Thus, a heritability, $h^{2}$, of zero indicates no genetic contributions to the trait, whereas $h^{2}=1$ implies that all trait variation between people is explained by genetic factors.

Heritability has been widely adapted in the context of brain imaging. Twin studies of brain structure have shown strong genetic influences over global properties such as the total grey and white matter volume 
\citep{Baare2001,Bohlken2014,Wright2002} and whole-brain microstructural integrity as quantified using fractional anisotropy \citep{Bohlken2014}. More specifically, inter-hemispheric tracts have been shown to demonstrate increased heritability \citep{Shen2014,Sudre2017} compared to more local intra-hemispheric connections. Paralleling these structural data, genetic influences have also been found using various fMRI measures of functional connectivity \citep{Colclough2017} \citep{Fu2015,Glahn2010,Sudre2017} and specific topological properties in structural \citep{Bohlken2014} and functional \citep{Fornito2011,Sinclair2015}  networks. Measures associated with both cost minimisation (modularity, clustering coefficient) as well as efficient information transfer (global efficiency, characteristic path length, small-worldness) demonstrate significant genetic contributions \citep{Sinclair2015,Bohlken2014}. These findings support the general assumption of genetic control over establishing effective communication while maintaining constraints on wiring cost. Integrating both cost minimization and communication efficiency into a single measure of network cost-efficiency, substantial genetic effects have been demonstrated in human functional connectivity networks \citep{Fornito2011}. Critically, brain regions showing significant heritability of cost-efficiency were located in the areas of paralimbic and association cortex, implying that network hubs comprise a genetically determined core of both structural and functional connectivity that facilitates efficient communication at relatively low cost.  

Whereas presenting an attractive approach for investigating genetic contributions, twin design poses some limitations. For example, the equal environment assumption underlying heritability models has been disputed based on several accounts \citep{Charney2017,Joseph2002}, since the shared prenatal environment differs between MZ and DZ twin pairs [e.g., MZ twins experience higher stress due to sharing a single placenta and outermost fetal membrane \citep{Corsello2010}] and MZ twins later in life are treated more similarly than DZ twins \citep{Joseph2002}. Moreover, differences in mitochondrial DNA and nuclear genome copy number variations mean that MZ twins are not $100\%$ genetically identical \citep{Bruder2008,Charney2017}. In addition, the estimation of genetic effects is contingent on population-level exposure. As a result, heritability estimates derived from twin studies are often regarded as inflated whereas the environmental contributions are likely to be consistently underestimated \citep{Joseph2002}. Despite these limitations, heritability is a useful starting point to quantify genetic contributions to a trait. 

\subsection{Structural DNA variation}

While heritability quantifies the proportion of trait variance attributable to genes, it does not identify which specific genetic variants may be involved. Such variants are typically identified using linkage and association analyses \citep{Thompson2013}. Linkage analyses exploit inheritance patterns within families in the aim of determining the co-segregation of the genotype and phenotype across individuals. Association analyses test for associations between the trait and allelic variation at different points scattered throughout the genome. The most comprehensive form of such analysis tests millions of markers scattered throughout the genome and is called a genome-wide association study (GWAS). Candidate gene studies, on the other hand, apply a hypothesis-driven approach by genotyping and testing markers that are hypothetically linked to a trait. 

Some of the first studies linking DNA variants to brain connectivity used a candidate gene approach by testing theoretically or experimentally derived hypotheses [for a review see \citet{Thompson2013}], and found associations with a range of brain connectivity phenotypes including, white matter microstructure \citep{Braskie2012,Chiang2011,Jahanshad2012b}, topological network properties \citep{Dennis2011}, and resting state functional connectivity \citep{Filippini2009,Trachtenberg2012,Westlye2011}. However, candidate gene studies have been called into question due to relatively high chance of false positives \citep{Sullivan2007},  and low replication rate potentially resulting from the population stratification \citep{Hutchison2004}. More recent GWAS analyses have identified variants related to structural connectivity \citep{Chiang2009,Jahanshad2013,Jahanshad2012a} and a broader set of imaging phenotypes \citep{Elliott2018}, however these studies require very large samples to achieve sufficient statistical power. Even within the ENIGMA consortium \citep{Thompson2014}, which was formed to address this issue, ensuring consistent quality control across multiple sites still remains a challenge. 

Considering the polygenic nature of most complex traits, with thousands of common variants contributing only small effect sizes, polygenic scores (PGS) allow quantification of the aggregate impact of multiple SNPs on a given trait \citep{Dudbridge2013}. The PGS is a weighted sum of SNP-specific effect-size estimates for a given trait derived from GWAS. By summing genetic contributions in a single score, it provides greater power for assessing relationships to imaging-derived phenotypes, at the expense of the ability to localize the specific variants driving the association. The general utility of PGS in the field of imaging genetics has been questioned due to its low predictive power, as PGS can potentially account for only a small proportion of the total phenotypic variance. Although a few large-scale analyses did not identify associations between the subcortical volumes and the genetic liability for psychiatric illness \citep{Franke2016,Reus2017}, several studies have found relationships between PGS for different psychiatric disorders and cortical gyrification \citep{Liu2016a}, functional connectivity \citep{Dezhina2018,Wang2017}, and longitudinal changes in white matter properties \citep{Alloza2018}. Some of those findings implicate known connectome hubs such as insula, cingulate and prefrontal cortex [for a review see \citet{Dezhina2018}] supporting the hypothesis that polygenic liability for psychiatric disorders influences functionally and structurally important regions. 

\subsection{Gene expression}

The influence of structural DNA variation on a given trait is mediated by the specific way in which those variants impacts gene function, but this impact is not immediately obvious when conducting GWAS or candidate gene studies. Particularly in GWAS, putatively causal variants are identified through statistical means, and may not be tagging the location of the actual causal variant. The development of brain-wide gene expression atlases \citep{Harris2010,Hawrylycz2012,Lein2007a} has opened opportunities for linking gene function to large-scale neural phenotypes measured across the entire brain [for a review see \citet{Fornito2019}]. The process of gene expression consists of multiple stages including transcription, when an unwound segment of DNA is read to produce messenger RNA (mRNA) and translation, when the resulting mRNA is used to synthesize a particular gene product. Gene expression is commonly inferred by quantifying levels of mRNA, which corresponds to transcriptional activity and is an indirect proxy for protein abundance \citep{Liu2016}. In species with high tissue availability gene expression can be measured at a cellular resolution using methods such as in-situ hybridization \citep{Lein2007a,Unger2010}, whereas bulk-tissue microarray \citep{Schulze2001} remains the most accessible method for high-throughput analysis in humans \citep{Hawrylycz2012}. Unlike heritability and structural DNA studies, which examine genetic contributions to phenotypic variability across individuals, atlas-based approaches investigate how spatial patterns of gene expression are related to spatial variations in brain structure or function. This can be achieved by rendering gene expression measures to evaluate (i) regional gene expression; (ii) correlated gene expression (CGE) between pairs of regions; or (iii) gene coexpression between pairs of genes. Each of those approaches addresses a distinct question: the aim of regional gene expression analyses is to identify associations between regional variations in gene expression and some regional property of the brain, such as node degree \citep{French2011}; CGE quantifies transcriptional coupling between regions across large swathes of the genome and can be related to pairwise properties of brain connectivity such as the presence or absence of a connection or the type of a connection they share \citep{Arnatkeviciute2018,Fulcher2016,Richiardi2015}; gene coexpression analyses focus on the correlations between the regional expression profiles of pairs of genes, thus evaluating which genes have similar expression patterns across the brain \citep{Forest2017}. 

Some of the first studies integrating atlas-based gene expression measures with brain connectivity demonstrated gene expression signatures to be predictive of both neuronal connectivity in \textit{C.elegans}
\citep{Kaufman2006} and regional connectivity in rodents \citep{Fakhry2015,Fakhry2015a,Ji2014}. Later regional gene expression patterns in the human brain have been shown to relate to a range of structural and functional properties including the specialisation of brain areas \citep{Anderson2018,Krienen2016,Parkes2017}, structural
\citep{Goel2014} and functional \citep{Cioli2014b,Forest2017,Richiardi2015,Vertes2016b} connectivity, as well as brain development \citep{Kirsch2016a,Whitaker2016a}. Building on evidence that the topological properties of hub connections are highly heritable \citep{Fornito2011}, Fulcher and Fornito investigated the transcriptional signatures of hub connectivity in the mouse brain \citep{Fulcher2016}. They demonstrated an elevated transcriptional coupling between spatially distributed hub regions that was primarily driven by genes regulating ATP synthesis and metabolism \citep{Fulcher2016}. These findings point to a molecular signature related to the high metabolic cost of hub connectivity and function \citep{Liang2013a,Tomasi2013}. Parallel evidence from fMRI in humans indicates that hubs with long-range connections that preferentially connect disparate neural systems may possess a comparable signature involving metabolism-related genes \citep{Vertes2016b} suggesting the existence of a conserved transcriptional signature associated with hub connectivity across species and scales. 

\section{Interim summary}

In summary, brain network architecture appears to be driven by the trade-off between minimizing the axonal wiring costs and promoting efficient, adaptive function \citep{Bullmore2012}. Converging evidence suggests that this balance is negotiated through the formation of a densely inter-connected rich-club – a putative communication backbone of the brain mediating a high proportion of communication traffic \citep{VandenHeuvel2011}. The emergence of hubs and rich-club connectivity across different species and scales points to common evolutionary pressures to maintain effective brain network communication at relatively low cost. Such pressures may be at least partly mediated through genes. Despite a growing body of literature demonstrating the genetic influences over brain connectivity, as quantified using heritability, structural DNA analysis [see \citet{Thompson2013} for a review], and brain-wide gene expression [see \citet{Fornito2019} for a review] only a limited number of studies have explicitly focused on the genetic basis of hubs and rich clubs. 

\section{Aims and overview of the thesis}

Following the initial evidence from twin study in humans suggesting a genetic contribution to cost-efficient properties of hub connectivity \citep{Fornito2011}, and the unique transcriptional signature of hub connectivity that may be conserved across mouse and human \citep{Vertes2016b,Fulcher2016}, the broad aim of this thesis is to comprehensively investigate genetic contributions to hub connectivity across different scales and levels of genetic influence. In particular, this thesis focuses on analyses of the connectomes of the nematode \textit{C.elegans} and the human. This thesis consists of six chapters, including three peer-reviewed manuscripts (two published, one under review). 

\textit{Chapter II} presents a peer-reviewed paper investigating hub connectivity and gene expression in the \textit{C.elegans} connectome. The goal of this work was to determine whether the transcriptional signature of hub connectivity identified in the mesoscale connectome of the mouse \citep{Fulcher2016} was also identified in the microscale connectome of the worm. The study combines a broad range of publicly available data and demonstrates that, like the mouse, connections between hub neurons of the worm show the most similar patterns of gene expression, and that this similarity is driven by genes involved in glutamatergic and cholinergic signalling, and other communication processes. It also shows that the increased gene expression similarity between hub neurons cannot be explained by a range of neuronal properties such as their subtype, separation distance, chemically secreted neurotransmitter, birth time, pairwise lineage distance, or the topological module affiliation. Instead, this coupling is linked to the functional designation of most hubs as command interneurons, a specific class of interneurons that regulates locomotion. 

\textit{Chapter III} provides a review of the current research investigating the transcriptional signatures of hub connectivity in neural networks across different species and scales including the findings in the \textit{C.elegans} connectome presented in chapter II. This manuscript was written as an invited article and currently is under review. It focuses specifically on the link between gene expression and hub connectivity and sets the stage for later chapters, that primarily investigate the genetic properties of hub connectivity in the human brain.

\textit{Chapter IV} entails a peer-reviewed paper focusing on methodological choices involved in integrating publicly available Allen Human Brain Atlas (AHBA) with neuroimaging. A growing number of studies is using the AHBA to map gene expression correlates of various brain imaging phenotypes, but inconsistencies in methods used for processing the AHBA data pose problems for reproducibility. This paper outlines the seven-step analysis pipeline for relating AHBA brain-wide transcriptomic and neuroimaging data and compares the influences of different processing choices. The aim of this work is to take a first step towards standardized analysis pipelines that can be routinely employed throughout the field. It is a necessary precursor for the analyses of Chapter V.

\textit{Chapter V} presents an extensive investigation of the genetic correlates of hub connectivity in the human brain. First, the genetic control over the structural brain networks is quantified through the edge-wise, connectome-wide heritability analysis. Second, effect of the structural genetic variation on hub connectivity is examined using polygenic scores for a range of psychiatric disorders and IQ. Finally, the transcriptional correlates of hub connectivity are investigated using AHBA gene expression data. The findings indicate that the strength of connectivity between hubs is the most highly heritable, is related to polygenic liability for mental illness and high IQ, and is characterized by the same transcriptional signature observed in mouse and \textit{C.elegans}; namely elevated transcriptional coupling predominately driven by metabolic genes. Together, these findings establish a link between molecular function and large-scale connectome organization across different species and modalities and demonstrate that the genetic influences on neuronal connectivity converge on brain network hubs. The general discussion, limitations and the directions for future research are presented in \textit{Chapter VI}. 



